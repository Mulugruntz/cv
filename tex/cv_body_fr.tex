% LaTeX file for resume 
% This file uses the resume document class (res.cls)

\documentclass{res}
\usepackage{enumitem}
\usepackage{tabto}
\usepackage{fontspec}
\usepackage{amssymb}
\usepackage{eurosym}
\usepackage[Symbols]{ucharclasses}
\topmargin=-0.5in  % start text higher on the page
\setlength{\textheight}{10.7in} % increase text height to fit resume on 1 page
\setlist[itemize,1]{leftmargin=\dimexpr 26pt-.7in} % to change the tab size globally
\addtolength{\textwidth}{0.8in}
\addtolength{\hoffset}{-0.5in}

\usepackage{fontawesome5}
\newfontfamily{\mysymbolfont}{Symbola}
\newfontfamily{\mymainfont}{DejaVu Sans}
\setTransitionsForSymbols{\mysymbolfont}{}

\begin{document}  
\ifisanon
	\name{CV ANONYME}
	\address{Des sections ont \'et\'e enlev\'ees. Elles sont indiqu\'ees par un \textit{(Anonyme)}}
\else
	\name{SAMUEL GIFFARD}

	\address{ \faMapMarker*~Pietersbergweg 53-B, 1105 BM, Amsterdam, Pays-Bas ~~ \faPassport~Fran\c{c}ais \\ \faAt~samuel@giffard.co ~~ \faMobile*~+33 (0) 7-69-69-12-02 ~~ \faMobile*~+31 (0) 6-8348-5505 ~~ \faWhatsapp~+32 (0) 4-84-14-10-33 \\ \\ \textbf{\faPython~Formateur, Ing\'enieur \& \'Evang\'eliste Python}  }
\fi

\begin{resume}
 
\section{\faGraduationCap~FORMATION}
\begin{itemize} 
\item[] 2011-2012 \tabto{2cm} MSc Computer Science \hfill University of Edinburgh\footnote{~UoE est 17$^{\grave{e}me}$ au classement mondial 2013. \`A titre de comparaison, University of California, Berkeley (UCB) est 25$^{\grave{e}me}$, tandis que les deux meilleures \'ecoles fran\c{c}aises, ENS Paris et Polytechnique sont respectivement 28$^{\grave{e}me}$ et 41$^{\grave{e}me}$. Source : \texttt{http://goo.gl/t78Vdc}} -- \'Ecosse 
	\begin{itemize}
		\item[+] \textbf{MSc Computer Science with Distinction} in Computer Systems, Software Engr. and High Perf. Computing
		\item[+] \textit{trad.} \textbf{Master Informatique} \`es Syst\`emes Informatiques, Ing\'enierie du Logiciel et Informatique de Haute Performance, \textbf{obtenu avec la Mention}\footnote{~La Mention est unique, il n'existe pas de paliers. \'Equivaut \`a un ``first-class''.}
	\end{itemize}
	
\item[] 2009-2011 \tabto{2cm} BSc (Hons) Computer Science \hfill Robert Gordon University -- \'Ecosse  
	\begin{itemize}
		\item[+] \textbf{First-class Bachelor wih Honours} in Computing for Internet and Multimedia
		\item[+] \textit{trad.} \textbf{Mention Tr\`es Bien}, Bac+4 en Informatique pour l'Internet et le Multim\'edia
	\end{itemize}
	
\item[] 2008-2009 \tabto{2cm} Licence 3 Informatique Fondamentale \hfill Toulouse III (Universit\'e Paul Sabatier) -- France

\item[] 2007-2008 \tabto{2cm} \textbf{DUT Informatique} \hfill IUT ``A'' -- Toulouse III -- France
	\begin{itemize}
		\item[+] Obtenu en une seule ann\'ee (``Ann\'ee Sp\'eciale'')
	\end{itemize}

\item[] 2006-2007 \tabto{2cm} CPGE TSI \hfill Lyc\'ee Jean-Baptiste Dumas -- Al\`es -- France
	\begin{itemize}
		\item[+] Classe Pr\'eparatoire aux Grandes \'Ecoles, Physique, Math\'ematiques et \'Electronique
	\end{itemize}
	
\item[] 2006 \tabto{2cm} Baccalaur\'eat STI G\'enie \'Electronique \hfill Lyc\'ee Antoine Bourdelle -- Montauban -- France
	\begin{itemize}
		\item[+] \textbf{Mention Bien} ($\sim$ top 10\%\footnote{~Source : \texttt{http://media.education.gouv.fr/file/48/0/5480.pdf} (p.4)})
	\end{itemize}
\end{itemize}


\section{\faMedal~R\'ECOMPENSES}
\begin{itemize}
	\item[] 18 avril 2018 \tabto{4cm} Python Challenge Europe 2018 \textbf{Winner Pro} (par Cisco, EDITx, ECAM)
	\item[] 15 juillet 2011 \tabto{4cm} \textbf{Major} de la promotion 2010 - 2011
	\item[] 1$^{er}$ juillet 2011 \tabto{4cm} \textbf{Nexen Prize} d\'ecern\'e pour le meilleur accomplissement acad\'emique
\end{itemize}


\section{\faLayerGroup~COMP\'ETENCES}
	\begin{itemize}
		\item[] \faLanguage~\textbf{Langues}
		\begin{itemize}
                 \item[+]  \textbf{Fran\c{c}ais} : langue maternelle, \textbf{Anglais} : bilingue, \textbf{Espagnol} : $\approx$B1, \textbf{Flamand} : $\approx$A2, \textbf{Japonais} : $\approx$A1
		\end{itemize}	
	\end{itemize}
	
	\begin{itemize}
	\item[] \faLaptopCode~\textbf{Informatique}
		\begin{itemize}
                  \item[+] \textbf{Concepts}: Architecture, Multim\'edia, S\'ecurit\'e, Bases de donn\'ees \textit{Relationnelles, Distribu\'ees \& Concepts avanc\'es}, Informatique de Haute Performance, Orient\'e Objet, Complexit\'e, Hadoop, Int\'egration \& D\'eveloppement Continus
                  \item[+] \textbf{Logiciels}: Oracle, PyCharm, Eclipse, Xcode, GDB, Valgrind, Adobe (Dw, Ps, Sb, Fl), git, Jenkins
                  \item[+] \textbf{Langages}: \textbf{Python}, FastAPI, Django, \textbf{C/C++}, Linux BASH, MySQL, SQLite, \textbf{Oracle PL/SQL}, EdgeDB, Objective-C (iOS), JAVA, ARM, OpenGL (Glut), \LaTeX{}, JS, jQuery, PHP, HTML/CSS, Bubble.io
		\end{itemize}    
	\end{itemize}

	\begin{itemize}
	\item[] \faHandshake[regular]~\textbf{Leadership}
		\begin{itemize}
			      \item[+] Deux fois \'elu Pr\'esident de l'ASSERP. J'ai orchestr\'e la gestion d'une dizaine de clubs et de leurs pr\'esidents.
                  \item[+] Head Judge pour le jeu international \textit{Magic:~The~Gathering} \`a Nice -- France
                  \item[+] J'ai men\'e plusieurs groupes de projets vers leur r\'eussite
		\end{itemize}   
	\end{itemize}
	
	\begin{itemize}
	\item[] \faComments[regular]~\textbf{Communication}
		\begin{itemize}
									\item[+] Supervision de stagiaires \`a Amadeus pendant trois mois en Django et pendant six mois en Node.js / Aria Template.
                  \item[+] Arbitre pour \textit{MTG}, communication au sein de l'\'equipe et communication avec les joueurs. La plupart des tournois ont plus de 120 joueurs et durent plus de 14 heures
                  \item[+] Travail en contact client\`ele dans des Pizzerias, Restaurants, Salon de th\'e
                  \item[+] Stage en milieu scolaire : Technologie au coll\`ege, Montauban -- France

                  \item[+]  Ambassadeur pour des visites du campus \`a la RGU, Aberdeen -- \'Ecosse
		\end{itemize} 
	\end{itemize}

\pagebreak

\section{\faBriefcase~EXP\'ERIENCES PROFESSIONNELLES}

	\begin{itemize}
        \item[] \textbf{Depuis Mars 2023} \tabto{5cm} Ing\'enieur logiciel s\'enior (Python, Fedora, PG, MySQL) \hfill Amsterdam -- Pays-Bas
        \begin{itemize}
            \item[+] Membre de l'\'equipe Produit responsable de PostgreSQL / MySQL chez \textbf{Aiven}.
        \end{itemize}
        \item[] F\'ev. 2022 - F\'ev. 2023 \tabto{5cm} Ing\'enieur Logiciel (Python, FastAPI, Docker, Recrutement) \hfill Amsterdam -- Pays-Bas
		\begin{itemize}
			\item[+] Cr\'eation de biblioth\`eques, d'outils et de proc\'ed\'es au sein de la Plateforme Python de \textbf{Picnic} (Middleware).
			\item[+] Recrutement: j'ai interview\'e, \'evalu\'e et embauch\'e des talents Python, et grandement am\'elior\'e les processus.
		\end{itemize}
		\item[] Avril 2021 - D\'ecembre 2021 \tabto{5cm} Manager \& Ing\'enieur s\'enior (Py, PHP, n8n, R-Pi, Bubble) \hfill Kitaky\=ush\=u -- Japon
		\begin{itemize}
			\item[+] Manager de projet chez \textbf{HOA} pour du SaaS ciblant les Coworking Spaces. Conception, tuning BDD \& workflows backend. Analyses, \'evaluation de productivit\'e, formation, conseil et outils B2B (Slack, Jira, Confluence).
			\item[+] Projet de reconnaissance faciale en IoT (Raspberry Pi, Python, ThingsBoard, MQTT).
		\end{itemize}
		\item[] F\'evrier  2018 - Mai 2019 \tabto{5cm} DevOps (Python, Jenkins, Groovy, GCP, PowerShell, Docker) \hfill Gand -- Belgique
		\begin{itemize}
			\item[] Promotion de l'am\'elioration de la qualit\'e logicielle au sein d'\textbf{Around Media}.
			\begin{itemize}
				\item[+] Pipelines avec Jenkins \& GCE : 1200+ UE4 builds, releases, migration d'assets, clonage \& backups de VM.
				\item[+] Cr\'eation de \ifisanon \texttt{github.com/...} \textit{(Anonyme)}, \else \texttt{github.com/Mulugruntz/jam} \fi pour \'economiser mensuellement 500\euro+ / instance GCE.
			\end{itemize}
		\end{itemize}
		\item[] Avril 2017 - Janvier 2018 \tabto{5cm} Ing\'enieur Informaticien (Python, Django, MariaDB, Docker) \hfill Hasselt -- Belgique
		\begin{itemize}
			\item[] D\'evelopeur MVP pour \textbf{Unleashed} (t\'el\'ecoms mobile) dans un environement Kanban + Squad.
		\end{itemize}
		\item[] \textbf{Depuis Juillet 2016} \tabto{5cm} Auto-entrepreneur Python \hfill Nice -- France
		\begin{itemize}
			\item[+] Projet priv\'e pour march\'es financiers. Missions Python \& Django. Formateur Python pour \textbf{AlterWay}, St Cloud.
		\end{itemize}
		\item[] F\'evrier 2016 - Juin 2016 \tabto{5cm} Ing\'enieur Informaticien (Java, Python, R-Pi, Linux) \hfill Sophia Antipolis -- France
		\begin{itemize}
			\item[] Projet Lightning chez \textbf{PwC} (IoT \& B2B, Raspberry Pi), par le biais de SII. Cr\'eation de :
			\begin{itemize}
				\item[+] Couche de persistence g\'en\'erique, appel\'ee en REST et de son impl\'ementation Json. Java / ServiceMix / Karaf.
				\item[+] Agent Syst\`eme dialoguant via XMPP avec une grammaire d\'efinie par modules. Python / Shell / smx / init.d.
				\item[+] Python (ex. DTMF \& Morse au-dessus de SoX). Front-End (HTML, Bootstrap \& uikit, Angular 1.5).
			\end{itemize}
		\end{itemize}
		\item[] Novembre 2015 - F\'evrier 2016 \tabto{5cm} Ing\'enieur Informaticien (PHP, Java, C++, Linux) \hfill Sophia Antipolis -- France
		\begin{itemize}
			\item[] Projet via Antyas chez \textbf{Airbus D\&S} class\'e secret D\'efense autour du traitement d'image des satellites SPOT.
		\end{itemize}
		\item[] Octobre 2012 - Juillet 2015 \tabto{5cm} Ing\'enieur Informaticien (C++, Python, Sqlite, Oracle)  \hfill Sophia Antipolis -- France
		\begin{itemize}
			\item[] \'Equipe de R\'eplication des Donn\'ees, d\'epartement Middleware d'\textbf{Amadeus}. Ax\'e R\&D, j'ai travaill\'e sur :
			\begin{itemize}
				\item[+] l'am\'elioration de performances g\'en\'erales (r\'eseau, bases de donn\'ees, algorithmie).
				\item[+] la r\'eplication de donn\'ees \`a travers des milliers de n\oe{}uds. Donn\'ees statiques, ou extr\^emement dynamiques.
				\item[+] la conception et l'impl\'ementation de types abstraits cohérents pour relier Oracle $\rightleftarrows$ C++ $\rightleftarrows$ Sqlite $\rightleftarrows$ Python.
				\item[+] divers scripts pour prototyper, acc\'el\'erer ou automatiser des t\^aches : stockage~/ r\'ecup\'eration~/ backup\textellipsis
				\item[+] l'export de donn\'ees depuis nos caches vers deux it\'erations successives de XMLs, en g\'en\'erant des XSLTs et des XSDs \`a partir de meta-XSLTs. Cela nous a permis de publier du XML avec diff\'erentes versions de schemas.
				\item[+] la conception et l'impl\'ementation d'un algorithme de Hepburn modifi\'e pour la translitt\'eration du japonais.
				\item[+] la d\'ecommission de commandes de terminaux depuis TPF (IBM) vers une nouvelle architecture en C++.
			\end{itemize}
		\end{itemize}
		\item[] Mai 2012 - ao\^ut 2012 \tabto{5cm} Projet de MSc (Python/Django, JS, HTML5/CSS3, \LaTeX{}) \hfill Edinburgh -- \'Ecosse
		\item[] \'Et\'e 2011 \tabto{5cm} Freelance (ind\'ependant) (PHP, JS, HTML5/CSS3) \hfill Montauban -- France
		\begin{itemize}
			\item[] Cr\'eation de \ifisanon \texttt{www....com} \textit{(Anonyme)} \else \texttt{www.qualigraphe.com} \fi et de son CMS modulaire sur mesure. Visuels pourvus par le client.
		\end{itemize}
		\item[] Octobre 2010 - mai 2011 \tabto{5cm} Projet d'Honours (PHP, JS, HTML/CSS) \hfill Aberdeen -- \'Ecosse
		\item[] F\'evrier - mai 2009 \tabto{5cm} Stage d'observation en milieu scolaire (Technologie au coll\`ege)\hfill Montauban -- France
		\item[] Juin - ao\^ut 2008 \tabto{5cm} Stage (PHP, JS, HTML/CSS) \hfill Toulouse -- France
		\item[] Octobre - Novembre 2007 \tabto{5cm} Maintenance Informatique \hfill Toulouse -- France
	\end{itemize}
		
\section{\faMagic~DIVERS}
	\begin{itemize}
		\item[] Permis B fran\c{c}ais
		\item[] \textbf{TOEFL} iBT (25 juin 2011): \textbf{107/120} (Lecture 23/30, \'Ecoute 27/30, Parl\'e 29/30, \'Ecrit 28/30)		
		    \end{itemize} 
 
\section{\faGamepad~HOBBIES}
	\begin{itemize}
		\item[] D\'ev, halt\'erophilie (5 ans), jeu de Go, MahJong MCR, voyages \& d\'ecouvertes, p\^atisserie, auto-stop, \'echanges culturels
	\end{itemize}

\section{\faUserCheck~R\'EF\'ERENCES (en anglais)}
\ifisanon 
    \begin{itemize}
			\item[] R\'ef\'erences disponibles sur demande. Quand l'anonymat sera lev\'e, pensez \`a les demander. \textit{(Anonyme)}
    \end{itemize}
\else
		\begin{itemize}
			\item[] Professor Fan Wenfei \tabto{4cm} \texttt{wenfei -chez- inf.ed.ac.uk} \hfill \texttt{http://homepages.inf.ed.ac.uk/wenfei}
			\item[] Dr. Niccolo Capanni \tabto{4cm} \texttt{niccolo -chez- capanni.co.uk} \hfill \texttt{http://www.linkedin.com/in/capanni}
		\end{itemize}
\fi
\end{resume}

\end{document}